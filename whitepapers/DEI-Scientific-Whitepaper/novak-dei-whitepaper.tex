\documentclass{article}
\usepackage{graphicx}

\documentclass[11pt]{article}

% --------------------------------------------------
% Packages
% --------------------------------------------------
\usepackage{amsmath, amssymb}
\usepackage{hyperref}
\usepackage{graphicx}
\usepackage{geometry}
\usepackage{float}
\usepackage{titlesec}

\geometry{
  a4paper,
  margin=1in
}

\titleformat{\section}{\large\bfseries}{\thesection}{1em}{}
\titleformat{\subsection}{\normalsize\bfseries}{\thesubsection}{1em}{}

% --------------------------------------------------
% DOCUMENT
% --------------------------------------------------
\begin{document}

\begin{center}
{\LARGE \textbf{A Scientific Examination of a New Class of Deterministic Execution-Integrity Primitives}\\[6pt]
{\large Matthew Novak --- 2025}
\end{center}

\begin{abstract}
This whitepaper identifies and formally characterizes a new family of cryptographic
primitives, forensic constructs, and execution-governance mechanisms collectively
referred to as the \textit{NOVAK Deterministic Execution Integrity Framework}. Unlike prior art in
digital signatures, attestations, or blockchain consensus, the NOVAK system introduces
a \textit{pre-action verification} model enabling deterministic validation of Rule, Data,
and Output before execution.

The paper examines five foundational primitives discovered in the NOVAK framework:
Proof-Before-Action (PbA), Hash-Verified Execution Trace (HVET), Execution Identity
Receipt (EIR), Recursive Global Audit Chain (RGAC), and the Deterministic Safety Gate.
These primitives, together with a novel cryptographic-forensic visualization layer,
constitute a distinct and previously unseen integrity architecture.

This work argues that NOVAK establishes a new scientific discipline:
\textbf{Deterministic Execution Integrity (DEI)}.
\end{abstract}
\newpage

\section{Introduction}

Modern systems rely on post-hoc mechanisms such as logging, digital signatures,
attestations, and blockchain-based consensus. These approaches cannot prevent
rule-evasion, silent manipulation, or post-output tampering.

The NOVAK system introduces the cryptographic requirement of \textit{Proof-Before-Action}
(PbA), meaning no computation proceeds until its governing rules, inputs, and expected
outputs are cryptographically registered and validated.

PbA represents a structural inversion of the modern security model and forms the core
discovery of this research.

\newpage
\section{Discovery 1: Proof-Before-Action (PbA)}

\subsection{Definition}

A Proof-Before-Action primitive canonically hashes the governing rule $R$, data $D$,
output $O$, and timestamp $t$ into a single deterministic tuple:

\[
    \text{HVET} = H(R \parallel D \parallel O \parallel t)
\]

The system must produce this proof \textit{before} executing the action.

PbA is neither a signature nor a log; it is an execution prerequisite. No existing
cryptographic or regulatory standard defines this behavior, making PbA an original
contribution.

\newpage
\section{Discovery 2: Hash-Verified Execution Trace (HVET)}

\subsection{Definition}

HVET binds rule identity, data payload, output state, and timestamp into one immutable,
audit-aligned fingerprint:

\[
    \text{HVET} = \text{SHA-256}(H_R \parallel H_D \parallel H_O \parallel t)
\]

A single-bit change in any component produces a completely different execution trace.

HVET differs from classical logs and attestations by being both pre-execution and
post-verifiable. Its dual use as a forensic visualization seed (Section~\ref{forensic})
is also novel.

\newpage
\section{Discovery 3: Execution Identity Receipt (EIR)}

The Execution Identity Receipt is a self-contained artifact representing an execution
event, including:

\begin{itemize}
    \item rule hash
    \item data hash
    \item output hash
    \item HVET
    \item timestamp
    \item PS-X compliance state
    \item RGAC anchor
\end{itemize}

No prior structure combines legal compliance, cryptographic binding, and visual
tamper-evidence in a single receipt format.

\newpage
\section{Discovery 4: Recursive Global Audit Chain (RGAC)}

RGAC is a deterministic audit mechanism defined as:

\[
A_{n+1} = H(A_n \parallel \text{EIR}_n)
\]

It is:

\begin{itemize}
    \item globally consistent,
    \item tamper-evident,
    \item non-consensus-based,
    \item lightweight,
    \item infinitely extensible.
\end{itemize}

RGAC is not a blockchain, Merkle tree, or append-only log. It is a new audit-chain
structure with zero distributed overhead.

\newpage
\section{Discovery 5: Deterministic Safety Gate}

The Safety Gate enforces:

\[
\text{Action allowed} \Longleftrightarrow \text{PbA valid}.
\]

It ensures that no system can take an action without cryptographic verification of rule,
data, and output correctness. This extends cryptographic guarantees into runtime safety.

\newpage
\section{Discovery 6: Cryptographic-Forensic Visualization Layer}
\label{forensic}

NOVAK introduces the first HTML-only treasury-grade forensic visualization system.
Elements are fully deterministic and HVET-seeded:

\begin{itemize}
    \item hash-derived guilloché patterns
    \item holographic dispersion fields
    \item fractal forensic substrate
    \item microtext halo containing L0--L15 + PL-X + PS-X
    \item triquetra security weave
    \item PS-X corner seals
    \item tamper-evident microprint border field
\end{itemize}

These elements historically require specialized printing machinery; here they are
generated algorithmically and reproducibly from the execution hash.

\newpage
\section{Discovery 7: A New Field --- Deterministic Execution Integrity (DEI)}

Together, PbA, HVET, EIR, RGAC, and the Safety Gate define a new execution model
that:

\begin{itemize}
    \item validates correctness before action,
    \item ensures immutability of execution state,
    \item binds legal, forensic, and cryptographic guarantees into one mechanism,
    \item operates independently of consensus systems.
\end{itemize}

We propose the formal term \textbf{Deterministic Execution Integrity (DEI)} for this field.

\newpage
\section{Conclusion}

The NOVAK discoveries constitute:

\begin{itemize}
    \item a new cryptographic primitive (PbA),
    \item a new canonical execution fingerprint (HVET),
    \item a new universal receipt format (EIR),
    \item a new deterministic audit chain (RGAC),
    \item a new safety enforcement model,
    \item a new HTML-based forensic visualization system.
\end{itemize}

Based on comparison with existing standards and published literature, NOVAK represents
a new class of deterministic integrity architecture and a foundational contribution to secure
automation, regulatory technology, and applied cryptography.

\end{document}
